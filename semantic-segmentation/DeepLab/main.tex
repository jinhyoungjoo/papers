\documentclass[10pt,twocolumn,letterpaper]{article}

\usepackage[pagenumbers]{cvpr}

\usepackage{graphicx}
\usepackage{amsmath}
\usepackage{amssymb}
\usepackage{booktabs}
\usepackage[pagebackref,breaklinks,colorlinks]{hyperref}
\usepackage[capitalize]{cleveref}
\crefname{section}{Sec.}{Secs.}
\Crefname{section}{Section}{Sections}
\Crefname{table}{Table}{Tables}
\crefname{table}{Tab.}{Tabs.}

\begin{document}

\title{Paper Review and Notes For\\DeepLab: Semantic Image Segmentation with\\Deep Convolutional Nets, Atrous Convolution, and Fully Connected CRFs}

\author{Jin Hyoung Joo\\ {\tt\small hyoungjoo.j@gmail.com} }
\maketitle

\begin{abstract}
    This paper \cite{DeepLab}
\end{abstract}

\section{Key Points}
\subsection{Proposed Methods for Segmentation}
The challenges of image segmentation and the authors' proposed method of handling
each problem can be seen as below.
\begin{itemize}
    \item{
        Image segmentation needs to predict the output with reduced feature resolutions.
        The authors used \emph{atrous convolutions} (convolution with upsampled filters)
        to help with this problem.
    }
    \item{
        Image segmentation needs to segment objects at multiple scales. The authors used
        \emph{atrous spatial pyramid pooling (ASPP)} to help with this problem.
    }
    \item{
        Deep convolution networks reduce the localization accuracy due to its invariance.
        The authors used \emph{fully connected Conditional Random Fields (CRF)s} to help
        with this problem.
    }
\end{itemize}

\subsection{Advantages of DeepLab}
DeepLab has the following advantages.
\begin{itemize}
    \item{DeepLab can operate at fast speeds.}
    \item{DeepLab achieves high accuracy.}
    \item{DeepLab has a simple structure.}
\end{itemize}

\section{Technical Details}
\subsection{Atrous Convolutions}
Reduced feature resolutions are due to repeated combinations of pooling and downsampling.
Deconvolutions help with this problem with the cost of additional memory and time.

Therefore to solve this problem, the last few pooling layers are removed and instead the
filters are upsampled in the subsequent convolution layers. The original image dimensions
are recovered using atrous convolutions and bilinear interpolation. Compared to regular
convolution with larger filters, atrous convolution allows us to effectively enlarge the
field of view of filters without increasing the number of parameters or the amount of
computation.
Atrous convolutions allow computing the responses of any layer at any desirable resolution.
This is done by either upsampling the kernel filter or by upsampling the input image by
inserting zeros.

\subsection{Atrous Spatial Pyramid Pooling}
One method of dealing with objects at multiple scales is to compute feature maps at
multiple scales and aggregating the output feature maps. This approach increases
the accuracy but it computationally expensive. Instead, DeepLab uses multiple parallel
atrous convolutions with different sampling rates.

\subsection{Fully Connected CRFs}
The model's ability to capture fine details are boosted by employing a fully-connected
Conditional Random Field. This method efficiently captures the fine edge details while
also catering for long range dependencies.

\section{Further Research}

{\small \bibliographystyle{cite-style} \bibliography{citations} }

\end{document}

% \begin{figure}[t]
%   \centering
%    \includegraphics[width=0.8\linewidth]{}
%    \caption{}
% \end{figure}

